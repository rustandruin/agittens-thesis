\documentclass[xcolor={svgnames,pdftex,dvipsnames,table},10pt]{beamer} %was usenames
\usefonttheme[]{serif} 
\usefonttheme{professionalfonts}
\usecolortheme[named=MidnightBlue]{structure}
\usetheme[height=7mm]{Rochester}
\setbeamertemplate{blocks}[rounded][shadow=true]
\useoutertheme{umbcfootline}
\useinnertheme{umbctribullets}
\useinnertheme{umbcboxes}
\setfootline{\insertshortauthor \quad \insertshorttitle \quad \insertshortdate \hfill \insertframenumber/\inserttotalframenumber}
\usepackage[utf8]{inputenc}
\usepackage{kerkis}
\usepackage{bm}
\usepackage{colortbl}
\usepackage[noend]{algpseudocode}
%\usepackage[scaled=0.875]{helvet}%
\renewcommand{\ttdefault}{lmtt}%

\usepackage{etoolbox}
\let\bbordermatrix\bordermatrix
\patchcmd{\bbordermatrix}{8.75}{4.75}{}{}
\patchcmd{\bbordermatrix}{\left(}{\left[}{}{}
\patchcmd{\bbordermatrix}{\right)}{\right]}{}{}

\renewcommand{\star}{*}

\DeclareMathOperator*{\argmin}{arg\,min}
\DeclareMathOperator{\gap}{gap}

\def\rf{\mathrm{F}}
\def\bigbar{\big\|}
\def\pinv{\dagger}
\def\transp{^T}
% 
\newcommand{\samats}[1]{\ensuremath{\mathbb{M}^{#1}_{\mathrm{sa}}}}
\newcommand{\s}{\sigma} % singular values
\newcommand{\asymO}[1]{\ensuremath{\mathop{\text{O}}\!\left(#1\right)}}
\DeclareMathOperator{\sgn}{sgn}

\newcommand{\norm}[1]{\ensuremath{\left\|#1\right\|}}
\newcommand{\supoverinfball}{\ensuremath{\max_{\infnorm{\mat{u}} = 1}}}
\newcommand{\supovertball}{\ensuremath{\max_{\norm{\mat{u}}_2 = 1}}}
\newcommand{\supoverqball}{\ensuremath{\max_{\qnorm{\mat{u}} = 1}}}
\newcommand{\eqqcolon}{=\mathrel{\mathop:}} %this may not be what I want, =:

\newcommand{\mat}[1]{\ensuremath{\mathbf{#1}}}
\renewcommand{\vec}[1]{\ensuremath{\mathbf{#1}}}
\newcommand{\e}{\ensuremath{\mathrm{e}}}
\newcommand{\E}{\ensuremath{\mathbb{E}}}
\newcommand{\Prob}[1]{\ensuremath{\mathbb{P}\left\{#1\right\}}}
\newcommand{\CondProb}[2]{\ensuremath{\mathbb{P}\left\{#1 \,\left|\, #2 \right.\right\}}}
\newcommand{\lnorm}[2]{\ensuremath{\left\| #2 \right\|_{L_{#1}}}}
\newcommand{\lipnorm}[1]{\ensuremath{\left\| #1 \right\|_\Pi}}
\newcommand{\R}{\ensuremath{\mathbb{R}}}
\newcommand{\Cfield}{\ensuremath{\mathbb{C}}}
\newcommand{\randcon}{\ensuremath{\Psi}}
\newcommand{\indicator}[1]{\ensuremath{\mathbb{1}\left\{#1\right\}}}

\newcommand{\xiNorm}[1]{\ensuremath{\left\|#1\right\|_\xi}}
\newcommand{\TNorm}[1]{\ensuremath{\left\|#1\right\|_2}}
\newcommand{\TNormS}[1]{\ensuremath{\left\|#1\right\|_2^2}}
\newcommand{\FNorm}[1]{\ensuremath{\left\|#1\right\|_\mathrm{F}}}
\newcommand{\FNormS}[1]{\ensuremath{\left\|#1\right\|_\mathrm{F}^2}}
\newcommand{\VTNorm}[1]{\ensuremath{\left\|#1\right\|_2}}
\newcommand{\VTNormS}[1]{\ensuremath{\left\|#1\right\|_2^2}}
\newcommand{\tracenorm}[1]{\ensuremath{\trace\left(#1\right)}}

\newcommand{\xinorm}[1]{\ensuremath{\big\|#1\big\|_\xi}}
\newcommand{\xinorms}[1]{\ensuremath{\big\|#1\big\|_\xi^2}}
\newcommand{\snorm}[1]{\ensuremath{\left\|#1\right\|_2}}
\newcommand{\snorms}[1]{\ensuremath{\big\|#1\big\|_2^2}}
\newcommand{\fnorm}[1]{\ensuremath{\left\|#1\right\|_{\mathrm{F}}}}
\newcommand{\fnorms}[1]{\ensuremath{\big\|#1\big\|_{\mathrm{F}}^2}}
\newcommand{\mcnorm}[1]{\ensuremath{\big\|#1\big\|_{\mathrm{mc}}}}
\newcommand{\mcnorms}[1]{\ensuremath{\big\|#1\big\|_{\mathrm{mc}}^2}}

\newcommand{\pnorm}[1]{\ensuremath{\left\|#1\right\|_p}}
\newcommand{\qnorm}[1]{\ensuremath{\left\|#1\right\|_q}}
\newcommand{\infnorm}[1]{\ensuremath{\left\|#1\right\|_\infty}}
\newcommand{\infonorm}[1]{\ensuremath{\left\|#1\right\|_{\infty\rightarrow 1}}}
\newcommand{\inftnorm}[1]{\ensuremath{\left\|#1\right\|_{\infty\rightarrow 2}}}
\newcommand{\onetwonorm}[1]{\ensuremath{\left\|#1\right\|_{1 \rightarrow 2}}}
\newcommand{\twoinfnorm}[1]{\ensuremath{\left\|#1\right\|_{2 \rightarrow \infty}}}
\newcommand{\smalltwoinfnorm}[1]{\ensuremath{\|#1\|_{2 \rightarrow \infty}}}
\newcommand{\infone}{\ensuremath{\infty\!\rightarrow\!\!1}}
\newcommand{\inftwo}{\ensuremath{\infty\!\rightarrow\!\!2}}
\newcommand{\infp}{\ensuremath{\infty\!\rightarrow\!p}}
\newcommand{\twoinf}{\ensuremath{2\!\!\rightarrow\!\!\infty}}
\newcommand{\colnorm}[1]{\ensuremath{\left\|#1\right\|_{\mathrm{col}}}}
\newcommand{\rownorm}[1]{\ensuremath{\left\|#1\right\|_{\text{\rmfamily{row}}}}}
\newcommand{\cutnorm}[1]{\ensuremath{\left\|#1\right\|_{\text{\rmfamily{C}}}}}

\newcommand{\F}{F}
\newcommand{\lambdamax}[1]{\ensuremath{\lambda_{\mathrm{max}}\left(#1\right)}}
\newcommand{\lambdamin}[1]{\ensuremath{\lambda_{\mathrm{min}}\left(#1\right)}}
\newcommand{\sigmamin}[1]{\ensuremath{\sigma_{\mathrm{min}}\left(#1\right)}}
\newcommand{\sigmamax}[1]{\ensuremath{\sigma_{\mathrm{max}}\left(#1\right)}}
\newcommand{\Isom}[2]{\ensuremath{\mathbb{V}_{#1}^{#2}}}
\DeclareMathOperator{\tr}{tr}
\DeclareMathOperator{\trace}{Tr}
\DeclareMathOperator{\rank}{rank}
\DeclareMathOperator{\var}{Var}
\newcommand{\trbrace}[1]{\ensuremath{\tr\left(#1\right)}}
\newcommand{\trexp}[1]{\ensuremath{\tr\exp\left(#1\right)}}
\newcommand{\const}[1]{\ensuremath{\mathrm{#1}}}

%\newtheorem{thm}{Theorem}[chapter]
%\newtheorem{prop}[thm]{Proposition}
%\newtheorem{lemma}[thm]{Lemma}
%\newtheorem{cor}[thm]{Corollary}
%\newtheorem{defn}[thm]{Definition}

%\theoremstyle{remark}
%\newtheorem{remark}[thm]{Remark}


% Bunch of stuff from the SRHT paper, need to eliminate redundancy and fold in
\newcommand{\Probab}[1]{\mbox{}{\mathbb{P}}\left(#1\right)}
\newcommand{\Exp}{\mbox{}{\mathbf{E}}}
\newcommand{\Expect}[1]{\mbox{}{\mathbb{E}}\left[#1\right]}
\newcommand{\Varnce}[1]{\mbox{}{\mathbf{Var}}\left[#1\right]}
\newcommand{\Trace }[1]{\mbox{}{\mathrm{Tr}}\left(#1\right)}
\newcommand{\diag  }[1]{\mbox{}{\textrm{diag}}\left(#1\right)}
\newcommand{\Sqrt  }[1]{\mbox{}\left(#1\right)^{1/2}}
\newcommand{\Qdrt  }[1]{\mbox{}\left(#1\right)^{1/4}}
\newcommand{\FNormB }[1]{\mbox{}\left\|#1\right\|_{\mathrm{F}}  }
\newcommand{\FNormBS}[1]{\mbox{}\left\|#1\right\|_{\mathrm{F}}^2}
\newcommand{\FNormF}[1]{\mbox{}\|#1\|_\mathrm{F}^4}
\newcommand{\FNormQ}[1]{\mbox{}\left\|#1\right\|_\mathrm{F}^4}
\newcommand{\TNormB }[1]{\mbox{}\left\|#1\right\|_2  }
\newcommand{\TNormBS}[1]{\mbox{}\left\|#1\right\|_2^2}
\newcommand{\TNormF}[1]{\mbox{}\|#1\|_2^4}
\newcommand{\TNormQ}[1]{\mbox{}\left\|#1\right\|_2^4}
\newcommand{\XNormB }[1]{\mbox{}\left\|#1\right\|_{\xi}  }
\newcommand{\XNormBS}[1]{\mbox{}\left\|#1\right\|_{\xi}^2}
\newcommand{\XNorm }[1]{\mbox{}\|#1\|_{\xi}  }
\newcommand{\XNormS}[1]{\mbox{}\|#1\|_{\xi}^2}
\newcommand{\PNorm }[1]{\mbox{}\|#1\|_{p}  }
\newcommand{\PNormS}[1]{\mbox{}\|#1\|_{p}^2}
\newcommand{\XNormQ}[1]{\mbox{}\left\|#1\right\|_{\xi}^4}
\newcommand{\VTNormQ}[1]{\mbox{}\left|#1\right|^4}
\newcommand{\VTTNorm }[1]{\mbox{}\left|#1\right|_2  }
\newcommand{\VTTNormS}[1]{\mbox{}\left|#1\right|_2^2}
\newcommand{\VTTNormQ}[1]{\mbox{}\left|#1\right|_2^4}
\newcommand{\MCNorm}[1]{\mbox{}\|#1\|_{1 \rightarrow 2}}
\newcommand{\MCNormS}[1]{\mbox{}\|#1\|_{1 \rightarrow 2}^2}
\newcommand{\INorm}[1]{\mbox{}\|#1\|_\infty}

\newcommand{\qedsymb}{\hfill{\rule{2mm}{2mm}}}

\newcommand{\dvc}{\ensuremath{d_{\small \sf VC}}}
\newcommand{\abs }[1]{\left|#1\right|}
\newcommand{\EE}[1]{\ensuremath{\mathbb{E}\left[#1\right] } }
\newcommand{\condE}[2]{\ensuremath{\mathbb{E}\left[#2 \,\left|\, #1 \right.\right]}}

\def\gammab{{\bm{\gamma}}}
\def\kappab{{\bm{\kappa}}}
\def\sig{{\bm{\Sigma}}}
\def\sigplus{{\bm{\Sigma}^{+}}}
\def\siginv{{\bm{\Sigma}^{-1}}}
\def\bet{{\bm{\beta}}}
\def\one{{\bm{1}}}
\def\exp{\ensuremath{\mathrm{exp}}}
\def\rank{\mathrm{rank}}
\newcommand{\stablerank}[1]{\operatorname{sr}\left(#1\right)}

%\def\cond{\hbox{\rm \mathbf{k}}}

\def\col{\hbox{\mathrm{col}}}
\def\ker{\hbox{\mathrm{ker}}}
\def\ahat{{\hat\a}}
\def\b{{\mathbf b}}
%\def\e{{\mathbf e}}
\def\expe{\mathrm{e}}
\def\q{{\mathbf q}}
\def\rb{{\mathbf r}}
%\def\s{{\mathbf s}}
%\def\u{{\mathbf u}}
\def\v{{\mathbf v}}
\def\x{{\mathbf x}}
\def\y{{\mathbf y}}
\def\z{{\mathbf z}}
\def\h{{\mathbf h}}
\def\d{{\mathbf \delta}}
\def\xhat{{\hat\x}}
\def\yhat{{\hat\y}}
\def\A{\matA}
\def\B{\matB}
%\def\C{\matC}
\def\Ahat{\hat\matA}
\def\Atilde{\tilde\matA}
\def\Btilde{\tilde\matB}
\def\Stilde{\tilde\matS}
\def\Utilde{\tilde\matU}
\def\Vtilde{\tilde\matV}
%\def\E{{\cl E}}
\def\G{{\cl G}}
\def\hset{{\cl H}}
\def\Q{{\bm{Q}}}
\def\U{{\bm{U}}}
\def\V{{\bm{V}}}
\def\win{\hat{\w}}
\def\wopt{\w^*}
\def\matAhat{\hat\mat{A}}
\def\matA{\mat{A}}
\def\matB{\mat{B}}
\def\matC{\mat{C}}
\def\matD{\mat{D}}
\def\matE{\mat{E}}
\def\matG{\mat{G}}
\def\matH{\mat{H}}
\def\matI{\mat{I}}
\def\matF{\mat{F}}
\def\matM{\mat{M}}
\def\matP{\mat{P}}
\def\matQ{\mat{Q}}
\def\matR{\mat{R}}
\def\matS{\mat{S}}
\def\matT{\mat{T}}
\def\matU{\mat{U}}
\def\matV{\mat{V}}
\def\matW{\mat{W}}
\def\matX{\mat{X}}
\def\matXopt{\mat{X}_{\mathrm{opt}}}
\def\matY{\mat{Y}}
\def\matZ{\mat{Z}}
\def\matSig{\boldsymbol{\Sigma}}
\def\matTh{\boldsymbol{\Theta}}
\def\matOmega{\boldsymbol{\Omega}}
\def\matPsi{\mat{\Psi}}
\def\matGam{\mat{\Gamma}}
\def\boldPi{\boldsymbol{\Pi}}
\def\w{{\mathbf{w}}}
\def\ein{{\cl E_{\text{\mathrm{in}}}}}
\def\eout{{\cl E}}
\def\scl{{\textsc{l}}}
\def\scu{{\textsc{u}}}
\def\phiu{{\overline{\phi}}}
\def\psiu{{\overline{\psi}}}
\def\phil{{\underbar{\math{\phi}}}}
%\DeclareMathSymbol{\Prob}{\mathbin}{AMSb}{"50}
%\DeclareMathSymbol{\Exp}{\mathbin}{AMSb}{"45}
\newcommand\remove[1]{}

\def\nnz{{ \mathrm nnz }}

%add footnotes to indicate support
\usepackage[absolute,overlay]{textpos} 
\newenvironment{support}[2]{% 
  \begin{textblock*}{\textwidth}(#1,#2) 
      \footnotesize\it\bgroup\color{black!50}}{\egroup\end{textblock*}}
			
\usepackage{tikz}
\usetikzlibrary{arrows,shapes,fit}
\tikzstyle{every picture}+=[remember picture]
\tikzstyle{na} = [baseline=-.5ex]
\everymath{\displaystyle}
\pgfdeclarelayer{background}
\pgfsetlayers{background,main}

\hypersetup{%
  pdftitle={Topics in Randomized Numerical Linear Algebra},%
  pdfauthor={Alex Gittens},%
  pdfsubject={PhD defense talk},%
  pdfkeywords={random matrix theory, randomized numerical linear algebra, low-rank approximation, sparsification, 
  dimension reduction}%
}
\hypersetup{plainpages    = false,
            pdfnewwindow  = false}
            %pdfpagemode   = FullScreen}
\hypersetup{colorlinks=true,urlcolor=blue}

% add bookmarks for easier navigations
\usepackage{bookmark}
\usepackage{etoolbox}
\makeatletter
\apptocmd{\beamer@@frametitle}{
  % keep this line to add the frame title to the TOC at the "subsection level"
  \addtocontents{toc}{\protect\beamer@subsectionintoc{\the\c@section}{0}{#1}{\the\c@page}{\the\c@part}%
        {\the\beamer@tocsectionnumber}}%
  % keep this line to add a bookmark that shows up in the PDF TOC at the subsection level
  \bookmark[page=\the\c@page,level=3]{#1}
  }%
  {\message{** patching of \string\beamer@@frametitle succeeded **}}%
  {\message{** patching of \string\beamer@@frametitle failed **}}%
\makeatother

%\setbeamertemplate{navigation symbols}{}
\setbeamercovered{dynamic}

\newtheorem*{thm}{Theorem}

% Titlepage info
\title{Topics in Randomized Numerical Linear Algebra}
\author[A. Gittens]{Alex Gittens}
\institute[Caltech]{%
Applied and Computational Mathematics \\
California Institute of Technology \\
\href{mailto:gittens@caltech.edu}{gittens@caltech.edu} \\
}

\date[May 2013]{May 31, 2013}

\begin{document}

\begin{frame}[plain]
\begin{support}{19mm}{85mm}

\end{support}
\titlepage
\end{frame}

\section{Introduction}
\begin{frame}{Randomized Numerical Linear Algebra}
\end{frame}

\begin{frame}
 We consider the performance of two classes of randomized low-rank approximants:
 \begin{itemize}
  \item \emph{Projection-based schemes}. We approximate $\matA$ by projecting
  it onto a random subspace of its range:
   $\matM = \matP_{\matA \matS} \matA$
   
  \item \emph{SPSD sketches}. Here $\matA$ is symmetric positive-semidefinite and we
  take $\matM = \matC \matW^\pinv \matC\transp,$ where 
  \[
   \matC = \matA \matS \quad \text{ and } \quad \matW = \matS\transp \matA \matS
  \]
  and $\matW^\pinv$ denotes the Moore-Penrose pseudoinverse of $\matW.$
 \end{itemize}

\end{frame}

\begin{frame}{Optimal rank-$k$ approximation}
Fix $\matA \in \R^{m \times n}$ with $m \geq n.$

The most accurate rank-$k$ approximation in the spectral and Frobenius norms
\[
 \matA_k = \underset{%
               \substack{%
                 \matM \in \R^{m \times n} \\ 
                 \operatorname{rank}(\tilde{\matA}) \leq k}}%
           {\operatorname{argmin}}  
           \|\matA - \matM\|_\xi, \quad \text{ for $\xi \in \{2, \mathrm{F}\}$}.
\]
can be computed in $\mathrm{O}(mn^2)$ time via the Singular Value Decomposition (SVD): if
\[
\mat{A} = \mat{U} \mat{\Sigma} \mat{V}\transp = \bbordermatrix{%
&^k \vspace{-0.75ex} & \!\!^{n-k}  \hspace{1ex}\cr
& \vspace{0.25ex} \mat{U}_1 \hspace{-2ex} & \mat{U}_2 
}
\bbordermatrix{%
& \vspace{-0.75ex} ^k &\!\!^{n-k} \hspace{1ex}\cr
& \mat{\Sigma}_1 & \cr
& \vspace{0.5ex} & \mat{\Sigma}_2 
}
\left[
\begin{matrix}
\mat{V}_1\transp \!\! \\
\mat{V}_2\transp \!\! 
\end{matrix}
\right]
\]
then $\matA_k = \matU_1 \matSig_1 \matV_1\transp.$

\end{frame}

\begin{frame}
 We measure the approximation accuracies relative to those of $\matA_k.$
 
 \begin{itemize}
  \item An approximation $\matM$ satisfies a \emph{relative-error bound} if
           \[
           \|\matA - \matM\|_\xi \leq (1 + \epsilon) \|\matA - \matA_k\|_\xi 
           \]
  for an $\epsilon > 0.$
  
  \item It satisfies an \emph{additive-error bound} if 
           \[
            \|\matA - \matM\|_\xi \leq \|\matA - \matA_k\|_\xi + \epsilon
           \]
  for an $\epsilon > 0.$ In this case, $\epsilon$ is called the \emph{additional error}.
 \end{itemize}

 
\begin{displaybox}{0.7\linewidth}
\parbox{\linewidth}{Goal: Find relative- and additive-error bounds for randomized low-rank approximation schemes.}
\end{displaybox}
 
\end{frame}

\begin{frame}{The target audience}
 Who is interested in these approximations, and what do they want?:
 \begin{itemize}
  \item The \emph{numerical linear algebra community} wants high quality approximations
  with very low failure rates and low communication cost.
  \item The \emph{machine learning community} wants approximations whose errors are on par
  with modeling inaccuracies and the imprecision of the data
  \item The \emph{optimization community} is interested in varying levels of quality.
  \item The \emph{theoretical computer science community} is interested in understanding the
  behavior of these algorithms, e.g. what is the optimal tradeoff between the error, failure rate,
  and the amount of arithmetic operations involved? How can communication cost be minimized?
 \end{itemize}
\end{frame}

\section{Overview of classical iterative methods}
\begin{frame}{Direct and iterative low-rank approximation methods}
 Classical \emph{direct} methods (e.g. SVD or rank-revealing QR factorizations) are not appropriate
 for large matrices:
 \begin{itemize}
  \item Require forming an expensive ($\mathrm{O}(mn^2)$ time) factorization of 
  $\matA$ first.
  \item Can densify intermediate matrices, so do not take advantage of sparsity.
 \end{itemize}
 
 Classical \emph{iterative} methods are more appropriate:
 \begin{itemize}
  \item They do not factorize $\matA.$
  \item They build low-rank approximations iteratively, and check for convergence
  \item They can take advantage of sparsity.
  \item They focus on recovering the singular spaces of the matrix.
 \end{itemize}
\end{frame}

\begin{frame}{Iterative method 1: Subspace iteration}

The top $k$-dimensional left invariant subspace of $\matA$ is the top $k$-dimensional eigenspace of $\matA\matA\transp.$
Subspace iteration captures this space by repeatedly applying $\matA\matA\transp$ to
a random set of vectors.
\[
\mathcal{R}((\matA\matA\transp)^q\matS) \rightarrow \mathcal{R}(\matU_1) 
\]
<Anim rotation of space as $B$ applied>

Speed of convergence is determined by $(\lambda_{k+1}(\matA)/\lambda_k(A))^q.$
\end{frame}

\begin{frame}
\algrenewcommand\algorithmicrequire{\textbf{Input:}}
 \algrenewcommand\algorithmicensure{\textbf{Output:}}
 \begin{algorithmic}[1]
  \Statex
  
 \State Let $\matY = \matA \matS.$
 \State Compute the QR decomposition $\matY = \matQ \matR.$
 \State Compute the SVD of $\matQ\transp \matA = \matW \tilde{\matSig} \tilde{\matV}\transp.$
 \State Set $\tilde{\matU} = \matQ \matW.$
\end{algorithmic}

\end{frame}

\begin{frame}{Iterative method 2: Lanczos iteration}
 Krylov subspace (Lanczos)
 methodology: capture top singular space, then use to approx singular values
 have guarantees on quality of singular space as function of mult. eigengap
 and iterations. 
 In practice, test for convergence, have numerical issues requiring
 complicated restarting/deflating etc. processes, efficiency depends on
 properties of the matrix (e.g. degenerate singular values cause issues)
 
 Separate issues of singular space from sing values
\end{frame}

\section{Fast projection-based low-rank approximation}
\begin{frame}{Projection-based low-rank approximations}
 <Illustrate>
 Capture top singular space of matrix, then project onto it
 Unlike the classical algorithms, the goal is to d
\end{frame}

\begin{frame}
 \algrenewcommand\algorithmicrequire{\textbf{Input:}}
 \algrenewcommand\algorithmicensure{\textbf{Output:}}
 \begin{algorithmic}[1]
  \Require{ an $m \times n$ matrix $\matA$ and an $n \times \ell$ matrix 
            $\matS,$ where $\ell$ is an integer in $[1,n].$ }
  \Ensure{matrices $\tilde{\matU}, \tilde{\matSig}, \tilde{\matV}$ 
          constituting the SVD of $\matP_{\matA\matS}\matA= \tilde{\matU} \tilde{\matSig} \tilde{\matV}\transp.$}	
  \Statex
  
 \State Let $\matY = \matA \matS.$
 \State Compute the QR decomposition $\matY = \matQ \matR.$
 \State Compute the SVD of $\matQ\transp \matA = \matW \tilde{\matSig} \tilde{\matV}\transp.$
 \State Set $\tilde{\matU} = \matQ \matW.$
\end{algorithmic}
\end{frame}

\begin{frame}{Comparison with classical iterative methods}
 Comparison with classical methods:
 how it's like Lanczos/Krylov-space w/o many iterations, or orthog iteration
 running time comparison (same-ish, but Lanczos complicated implementation 
 and number of iterations variable on matrix), communication cost comparison (better, 2ish passes)
 importance of error bounds to guiding understanding (b/c can't check error iteratively)
\end{frame}

\begin{frame}
 <Illust> examples of it working
\end{frame}

\begin{frame}
 History: , origi by Papadimitriou, popularized by Sarlos, Wolfe et al. use randomized projections 
\end{frame}

\begin{frame}
 Importance of oversampling, spectral decay, power method, type of randomness used
\end{frame}

\begin{frame}{The advantage of fast transformations}
 Advantage of subsampled fast orthogonal transforms when not using power method. E.g. SRHT (define) 
\end{frame}

\begin{frame}
introduce SRFT etc.
 Wolfe et al. realized fast transformation
 known results
 \end{frame}

\begin{frame}
Goal: to reduce the amount of oversampling needed to get good bounds
 Known results vs our results for SRHT (when one better than other)
 mention did work with Christos
\end{frame}

\begin{frame}
 Structural result at the basis of our result
 Idea: take advantage of spectral decay in the tangent
\end{frame}

\begin{frame}
 main ideas of proof: 
 spread out energy evenly across columns of matrix, connection w/ spectral norm
 preserve singular values via concentration argument
 preserve spectral norm
 matrix multiplication result
\end{frame}

\begin{frame}
 outline spectral norm result
\end{frame}

\begin{frame}
 outline Frobenius norm result
\end{frame}

\section{SPSD Sketches}
\begin{frame}
 problem of preserving positivity in addition to getting low-rank approximation
 could use above projection-based schemes
 alternative schemes which eliminate projection step (give spsd approx scheme)
\end{frame}

\begin{frame}
 How the two compare: 
 stable, low-error w/ two passes
 unstable in general, low-error w/ one-pass still higher than other 
 when take as many passes as stable, even lower error
 <Illust>
\end{frame}

\begin{frame}
Note sketching allows for both column-sampling and mixture-based schemes

In particular, SPSD sketch scheme incorporates Nystr\"om extensions, which use
 no information on matrix other than size, requires only column sampling
 <Illust>
\end{frame}

\begin{frame}
 introduce more particular examples of sketching schemes: gaussian, levscore, SRFT
\end{frame}

\begin{frame}
 explain leverage scores (better measure of nonuniformity)
 explain coherence (cruder)
\end{frame}

\begin{frame}
 Previous results on SPSD sketching (non-adaptive!):
 Nystrom, CD11
 give table
 our results asymptotically better
 comment not many other SPSD sketching schemes considered 
\end{frame}

\begin{frame}
 give table w/ error information
 argue all perform well in practice
\end{frame}

\begin{frame}
 Approach to analyzing SPSD sketch performance:
 First deterministic results
 Plug in stochastic for tangent
\end{frame}

\begin{frame}
 Key point: realize the factorization of CWC in terms of $A^{p- 1/2}$
 Note can replace $A^{1/2}$ with generalized Cholesky factorization
 Also explains why SPSD sketches with same number of iterations better
 than projection based approxs
\end{frame}

\begin{frame}
 derive spectral bound -- can relate directly to earlier deterministic bound
\end{frame}

\begin{frame}
 derive Frobenius bound -- need to do some more estimations
\end{frame}

\begin{frame}
 derive trace norm bound -- simple
\end{frame}

\begin{frame}
 Given established framework, note which quantities need bounding
 For leverage-based scheme, blah blah
 For gaussian-based scheme, blah blah
\end{frame}

\begin{frame}
state leverage based result
\end{frame}

\begin{frame}
 state gaussian based result
 compare to leverage-based
\end{frame}

\begin{frame}
 state SRFT-based result
 compare to gaussian/lev
\end{frame}

\begin{frame}
To get Nystr\"om result, need bound blah, observe
can be written as sum of rank one mats,
use matrix Chernoff bound
\end{frame}

\begin{frame}
 state and explain matrix Chernoff bound
\end{frame}

\begin{frame}
recall coherence
 state Nystr\"om result
 point out dependence on $\mu$
\end{frame}

\begin{frame}
 illustrate dependence on $\mu$ and sparsity of U1
\end{frame}

\begin{frame}
 illustrate optimality dimensional dependence in rel err bound for Nystrom
\end{frame}

\begin{frame}
 prove optimality of rel err bound for Nystrom
\end{frame}

\begin{frame}
 introduce stabilization algorithms:
 one use Tikhonov regularization, one truncates W
 one which I thought was due to me but actually due to WS01, one due to CD11
\end{frame}

\begin{frame}
 Illust errors of regularized SPSD sketches 
 argue that CD11 better in terms of norm reconstruction, but keep in mind
 down stream apps, is more violent, so might make sense to look at WS01 alg
\end{frame}

\begin{frame} 
 State result in CD11 for truncated W which applies only to Nystr\"om
\end{frame}

\begin{frame}
 State theorem for WS01 algorithm, and about condition numbers
 
 (what is a reasonable condition number for solving linear systems? this 
 would motivate choice of regularization parameter, and say whether this
 bound is useful)
\end{frame}

\begin{frame}
 Combine with some particular SPSD sketch to get guarantees for the
 regularized version
\end{frame}
% 
% \section{Sparsification}
% \begin{frame}{Randomized matrix sparsification}
% Switch gears.
%  Idea: Replace a dense matrix with a nearby sparse matrix, because the
%  algorithms for sparse matrices are less resource intensive. Decreases
%  arithmetic cost, not necessarily communication cost
% 
%  <Graphic here>
%  
%  Mention SVD times
% \end{frame}
% 
% \begin{frame}
%   \begin{displaybox}{0.7\linewidth}
%  Question: Given $\matA \in \R^{m \times n}$ and $\epsilon > 0,$ how
%  can we find a ``sparse'' matrix $\tilde{\matA}$ that satisfies $\|\matA - \tilde{\matA}\|_2 \leq \epsilon?$
% \end{displaybox} 
% 
% \begin{itemize}
%  \item (Achlioptas and McSherry) proposed 
%  \item (Hazan and Kale) proposed
%  \item (Zouzias and )
%  \item (Liberty )
%  \item (Nguyen et al)
%  
%  mention communication costs
% \end{itemize}
% 
% \end{frame}
% 
% \begin{frame}
%  Sparsification also has potential connections in graph theory because of connections
%  between graphs and matrices
%  
%  infty->1 norm connection
%  
%  cite Mahoney paper
% \end{frame}
% 
% \begin{frame}
%  Sparsification connections to variable selection,
%  infty->2 norm connection
%  
%  cite Tropp, etc.
% \end{frame}
% 
% \begin{frame}
%  Comment on fact that both infty->1 and infty->2 norms are NP-hard 
% \end{frame}
% 
% \begin{frame}
%  For all three norms: infty->1, infty->2, spectral, considering
%   properties of random matrices with zero mean and independent entries
%   
%   Strategy: estimate the expected norms, use concentration result to
%   show don't deviate much from expected norm
% \end{frame}
% 
% \begin{frame}
%  spectral norm argument, build on Latala's result (mention Seginer, etc.)
% \end{frame}
% 
% \begin{frame}
%  Infty->p norm lemma outline
% \end{frame}
% 
% \begin{frame}
%  Apply to infty->1 norm, mention optimality of bound
% \end{frame}
% 
% \begin{frame}
%  mention incomparability to latest matrix sparsification because best modeled
%  as sum of random matrices, not elment wise
%  
%  motivates consideration of eigenvalue bounds
% \end{frame}
% 
% 
% \section{Eigenvalue Bounds}
% \begin{frame}
%  sums of random matrices come up, e.g. in sparsification, nonasymptotic statistics bounds
%  
%  pose main question: how do the eigenvalues of sums of hermitian random matrices behave?
% \end{frame}
% 
% 
% \begin{frame}
%  what is known: extremal eigenvalue bounds, and asymptotic theory for ensembles
%  difficulty of min-max formulation of all eigenvalues
% \end{frame}
% 
% \begin{frame}
%  we build on the matrix laplace transform methodology (introduce it and talk about
%  classical tail bounds make it particularly useful)
% \end{frame}
% 
% \begin{frame}
%  extend mlt framework to extend to all eigenvalues
%  state Courant--Fischer theorem
% \end{frame}
% 
% \begin{frame}
%  outline argument for extension
% \end{frame}
% 
% \begin{frame}[Chernoff bounds]
%  Give chernoff bounds
% \end{frame}
% 
% \begin{frame}
%  Application: subsampling columns from orthonormal rows
%  mention coherence connection
% \end{frame}
% 
% \begin{frame}
%  figure of subsampling
% \end{frame}
% 
% \begin{frame}[Bernstein bounds]
%  Give Bernstein bounds
% \end{frame}
% 
% \begin{frame}
%  Give application: all eigenvalues of covariance matrix convergence rate
%  Note this gives more info than sharpest known nonasymptotic result (b/c it deals with
%  interior eigenvalue)
% \end{frame}
% 
% \begin{frame}
%  application: recall sparsification,
%  here's a result on convergence of all eigenvalues of sparsified matrix
% \end{frame}
% 
% 
% \begin{frame}{Problem Statement}
% 
% Let $\mat{x} \in \R^p$ be a zero-mean high-dimensional random vector. Information on the dependence structure of $\mat{x}$ is captured by the covariance matrix
% \[
% \mat{\Sigma} = \E \mat{x} \mat{x}^\star.
% \]
% The sample covariance matrix is a classical estimator for $\mat{\Sigma}:$
% \[
% \widehat{\mat{\Sigma}}_n = \frac{1}{n} \sum\nolimits_{i=1}^n \mat{x}_i\mat{x}_i^\star.
% \]
% 
% \begin{displaybox}{0.7\linewidth}
% \parbox{\linewidth}{How many samples of $\mat{x}$ are required so that $\widehat{\mat{\Sigma}}_n$ accurately estimates $\mat{\Sigma}?$}
% \end{displaybox}
% 
% \end{frame}
% 
% \begin{frame}{What is known}
% 
% Typically accuracy is measured in spectral norm.
% \vspace{1em}
% \begin{displaybox}{0.7\linewidth}
% \parbox{\linewidth}{%
% How many samples ensure that
% \[ \|\mat{\Sigma} - \widehat{\mat{\Sigma}}_n\|_2 \leq \varepsilon \|\mat{\Sigma}\|_2? \]
% }
% \end{displaybox}
% \pause
% \begin{itemize}
%     \item for \textcolor{OliveGreen}{log-concave} distributions $\Omega(p)$ samples suffice (Adamczak et al. 2011), 
% 		\pause
%     \item for distributions with \textcolor{OliveGreen}{finite fourth moments}, $\tilde{\Omega}(p)$ samples suffice (Vershynin 2011a), 
% 		\pause
% 		\item for distributions with \textcolor{OliveGreen}{finite $2+\varepsilon$ moments} that satisfy a regularity condition, $\Omega(p)$ samples suffice (Vershynin 2011b),
% 		\pause
%     \item for distributions with \textcolor{OliveGreen}{finite second moments}, $\Omega(p\log p)$ samples suffice (Rudelson 1999). 
% \end{itemize}
% \end{frame}
% 
% \begin{frame}{An observation}
% A relative spectral error bound,
% \[
% \|\mat{\Sigma} - \widehat{\mat{\Sigma}}_n\|_2 \leq \varepsilon \|\mat{\Sigma}\|_2,
% \]
% ensures recovery of the top eigenpair of $\mat{\Sigma},$ \ldots
% 
% \vspace{1em}
% \pause
% but does \emph{not} ensure the recovery of the remaining eigenpairs:
% \[
% |\lambda_k(\mat{\Sigma}) - \lambda_k(\widehat{\mat{\Sigma}}_n)| < \varepsilon \|\mat{\Sigma}\|_2 
% \]
% is not meaningful if $\lambda_k \ll \lambda_1.$
% 
% \vspace{1em}
% \pause
% Using known relative spectral error bounds, need O$(\varepsilon^{-2} \kappa(\mat{\Sigma}_\ell)^2 p)$ measurements to get relative error recovery of the top $\ell$ eigenvalues.
% \end{frame}
% 
% \begin{frame}{\dots and a question}
% 
% Maybe $\mat{\Sigma}$ has a decaying spectrum.
% What if we want accurate estimates of a few of its eigenvalues?
% \vspace{1em}
% \begin{displaybox}{0.7\textwidth}
% \parbox{\textwidth}{%
% How many samples ensure the top $\ell \ll p$ eigenvalues are estimated to relative accuracy,
% \[
% |\lambda_k(\mat{\Sigma}) - \lambda_k(\widehat{\mat{\Sigma}}_n)| \leq \varepsilon \lambda_k(\mat{\Sigma})?
% \]
% }
% \end{displaybox}
% 
% \vspace{1em}
% Do we really need O($p$) measurements to recover just a few of the top eigenvalues?
% 
% \end{frame}
% 
% \begin{frame}{A simplified result}
% 
% \begin{thm}
% Let the samples be drawn from a $\mathcal{N}(\mat{0}, \mat{\Sigma})$ distribution. Assume $\lambda_k$ decays sufficiently for $k > \ell$. If $\varepsilon \in (0, 1]$ and 
% \[
% n = \Omega(\varepsilon^{-2} \kappa(\mat{\Sigma}_\ell)^2 \ell \log p),
% \]
% then with high probability, for each $k=1,\ldots,\ell,$
% \[
% |\lambda_k(\widehat{\mat{\Sigma}}_n) - \lambda_k(\mat{\Sigma})| \leq \varepsilon \lambda_k(\mat{\Sigma}) 
% \]
% \end{thm}
% 
% \begin{itemize}
% \pause
% 	\item Sufficient decay is, (other conditions give other results)
% 	\[ \sum_{i > \ell} \lambda_i/\lambda_1 \leq C.
% 	\] 
% 	This is satisfied if, e.g., the tail eigenvalues, $k > \ell,$ correspond to spread-spectrum noise or decay like $\tfrac{1}{i^{(1+\iota)}
% 	}$ for some $\iota > 0.$
% 	\pause
% 	\item The approach generalizes to other subgaussian distributions.
% \end{itemize}
% 
% \end{frame}
% 
% \begin{frame}{More generally}
% Restrict, for each $k$, probability that $\hat{\lambda}_k$ under/overestimates $\lambda_k.$
% \begin{itemize}
% \pause 
% \item an upper bound on $\lambda_k$ 
% \[
% n = \frac{8}{3 \varepsilon^2} \tikz[baseline] \node[anchor=base,rounded corners,fill=OliveGreen!30] {$\kappa(\mat{\Sigma}_k) \frac{\tr{\mat{\Sigma}_k}}{\lambda_k}(\log k + \beta \log p)$}; \Rightarrow \Prob{\frac{\hat{\lambda}_k}{1-\epsilon} > \lambda_k} > 1- p^{-\beta}
% \]
% 
% \pause
% 	\item a lower bound on $\lambda_k$
% \begin{multline*}
% n = \frac{1}{32\varepsilon^2} 
% \tikz[baseline] \node[anchor=base,rounded corners,fill=BurntOrange!30] {$\frac{\big(\sum\nolimits_{i \geq k}\lambda_i \big)}{\lambda_k} \textstyle (\log (p-k+1) + \beta \log p)$}; \\
% \Rightarrow \Prob{\frac{\hat{\lambda}_k}{1+\varepsilon} < \lambda_k} > 1-p^{-\beta}.
% \end{multline*}
% 
% \pause 
% 
% \item Assuming decay
% \begin{center}
% 	\begin{tabular}{l >{\columncolor{OliveGreen!30}}c >{\columncolor{BurntOrange!30}}c}
% 	\multicolumn{1}{l}{} & \multicolumn{1}{l}{upper bound} & \multicolumn{1}{l}{lower bound} \\ 
% 		$\lambda_1$ & O$(\log p)$ & O$(\ell \log p)$ \\
% 		$\lambda_\ell$ & O$(\kappa^2(\mat{\Sigma}_\ell) \ell \log p)$ & O$(\kappa(\mat{\Sigma}_\ell) \log p)$ 
% 	\end{tabular}
% \end{center}
% 
% %\pause
% 	%\item a lower bound on $\lambda_k$
% %
% %\[
% 	%n = \frac{8}{3\varepsilon^2} 
% 	%\begin{tikzpicture}[baseline]
% 	%\node[anchor=base] (n2) 
% 	%{$\kappa(\mat{\Sigma}) \frac{\tr{\mat{\Sigma}}}{\lambda_p} (\log p + \delta)$}; 
% 		%
% %\begin{pgfonlayer}{background}
% %\node[fill=BurntOrange!40,draw=BurntOrange,fit=(n2)] {};
% %\end{pgfonlayer}
% %\end{tikzpicture}
% 	%\Rightarrow \Prob{\hat{\lambda}_p \leq (1-\varepsilon)\lambda_p} \leq \e^{-\delta}.
% %\]
% 	%\begin{itemize}
% 	 %\item boo
% 	%\end{itemize}
% 	%
% \end{itemize}
% 
% %\pause 
% %If the spectrum decays:
% %\begin{itemize}
%  %\item Can be as small as $\log^2 p$  \tikz[na] \node[coordinate] (l1) {};
% %\begin{tikzpicture}[overlay]
%  %\path[->,OliveGreen,thick] (l1) edge [out=90, in=-90] (n1);
% %\end{tikzpicture}
% %
% %\pause
% %
%  %\item Can be as small as $\kappa(\mat{\Sigma})^2 \log^2 p$  \tikz[na] \node[coordinate] (l2) {};
% %\begin{tikzpicture}[overlay]
%  %\path[->,BurntOrange,thick] (l2) edge [out=0, in=-90] (n2);
% %\end{tikzpicture}
% %\end{itemize}
% %
% \end{frame}
% 
% \begin{frame}{Proof sketch}
% 
% It suffices to show
% \[
%  \textstyle \Prob{\hat{\lambda}_k \geq (1+\varepsilon) \lambda_k } \quad \text{ and } \quad \Prob{\hat{\lambda}_k \leq (1-\varepsilon) \lambda_k}
% \]
% decay like $\mathrm{C} \exp(-\mathrm{c} n \epsilon^2)$ when $\epsilon$ is sufficiently small.
% 
% \pause
% \begin{enumerate}
% 		
% 	\item Reduce the probability of each case occuring to the probability that the norm of an appropriate matrix is large.
% 
% 	\pause
% 	
% 	\item Use matrix Bernstein bounds to establish the correct decay of these norms. 
% 
% 	\pause
% 	
% 	\item Take a union bound over the indices $k.$ 
%  \end{enumerate}
%  
% \end{frame}
% 
% % \begin{frame}{Reduction for $\hat{\lambda}_k \geq \lambda_k + t$}
% % 
% % Let $\mat{B}$ have orthonormal columns and span the bottom $(p-k+1)$-dimensional invariant subspace of $\mat{\Sigma}.$
% % 
% % \underline{Claim}
% % \[
% % \Prob{
% % \begin{tikzpicture}[baseline]
% % \node[anchor=base] (esteigenval) {$\hat{\lambda}_k$};
% % %\onslide<3->{\begin{pgfonlayer}{background}
% % %\node[fill=OliveGreen!40,draw=OliveGreen,fit=(esteigenval)] {};
% % %\end{pgfonlayer}}
% % \end{tikzpicture}
% % \geq 
% % \tikz[baseline]{\node[anchor=base] (eigenval) {$\lambda_k$};} + t 
% % } \leq
% % \Prob{
% % \tikz[baseline]{\node[anchor=base] (pinchedesteigenval) {$\lambda_1(\mat{B}^\star \widehat{\mat{\Sigma}}_n \mat{B})$}; }
% % \geq  
% % \tikz[baseline]{\node[anchor=base] (eigenvalexpansion) {$\lambda_1(\mat{B}^\star \mat{\Sigma} \mat{B})$};} + t }.
% % \]
% % 
% % \pause
% % \emph{Proof.}
% % 
% % By Courant--Fischer,
% % \[
% %  \lambda_k(\mat{\Sigma}) = \lambda_1(\mat{B}^\star \mat{\Sigma} \mat{B})
% % \]
% % 
% % \pause
% % and
% % \[
% % \lambda_k(\widehat{\mat{\Sigma}}_n) = \min_{\substack{ \mat{V} \in \C^{p \times (p-k+1)} \\ \mat{V}^\star\mat{V}=\mat{I}}} \lambda_1(\mat{V}^\star \widehat{\mat{\Sigma}}_n\mat{V}) \leq \lambda_1(\mat{B}^\star \widehat{\mat{\Sigma}}_n \mat{B}).
% % \]
% % 
% % \qed
% % \end{frame}
% 
% \begin{frame}{Using the reduction}
% Need to control RHS of 
% \[
% \Prob{\hat{\lambda}_k \geq \lambda_k + t }
% \leq
% \Prob{
% \lambda_1(\mat{B}^\star \widehat{\mat{\Sigma}}_n \mat{B}) \geq  
% \lambda_1(\mat{B}^\star \mat{\Sigma} \mat{B}) + t 
% }
% \]
% \pause
% Note:
% \begin{itemize}
% 	\item 
% $\lambda_1(\mat{B}^\star \hat{\mat{\Sigma}}_n \mat{B}) \rightarrow \lambda_1(\mat{B}^\star \mat{\Sigma} \mat{B}),$ and
% 	\item $\mat{B}^\star \hat{\mat{\Sigma}}_n \mat{B} = \sum\nolimits_i \mat{B}^\star \mat{x}_i\mat{x}_i^\star \mat{B}$ is a sum of independent random matrices.
% \end{itemize}
% \pause
% Use estimates of the matrix moments of the summands to quantify the convergence. 
% \begin{itemize}
% \pause
% \item If $\mat{g} \sim \mathcal{N}(\mat{0}, \mat{C})$, then for $m \geq 2,$
% \[
% \E(\mat{g}\mat{g}^\star)^m \preceq 2^m m!\,(\tr{\mat{C}})^{m-1} \cdot \mat{C}.
% \]
% \item Other subgaussian distributions satisfy similar relations. Can also substitute bounds on matrix moment generating functions, 
% \[
% \E \exp\left(\theta \mat{y}\mat{y}^\star \right) \preceq \mat{U}(\theta).
% \]
% \end{itemize}
% 
% \end{frame}
% 
% %\begin{frame}{Bernstein inequality}
%  %Consider the probability
% %\begin{multline*}
%  %\Prob{\lambda_1(\mat{B}^\star \widehat{\mat{\Sigma}}_n \mat{B}) \geq  \lambda_1(\mat{B}^\star \mat{\Sigma} \mat{B}) + t } \\
%  %= \Prob{\lambda_1\left(\sum_{i=1}^n \mat{B}^\star \mat{x}_i \mat{x}_i^\star \mat{B} \right) \geq n \lambda_1(\mat{B}^\star \mat{\Sigma} \mat{B}) + n t}.
% %\end{multline*}
% %
% %The real variable analogue,
% %\[
%  %\Prob{ \sum_{i=1}^n x_i \geq t },
% %\]
% %can be bounded, using Bernstein's inequality, in terms of the variances of the summands and a uniform bound on their magnitudes.
% %\end{frame}
% 
% \begin{frame}{Matrix Bernstein inequality }
%  We use a moment-based matrix analog of Bernstein's inequality.
% 
% \begin{thm}[Matrix Moment-Bernstein Inequality]
% Suppose self-adjoint matrices $\{\mat{G}_i\}$ have dimension $d$ and 
% \[ \E(\mat{G}_i^m) \preceq \frac{m!}{2} A^{m-2} \cdot \mat{C}_i^2 \quad \text{ for } m =2,3,4,\ldots.
% \]
% Set
% \[ \mu = \lambda_1\Big( \sum\nolimits_i \E\mat{G}_i \Big) \quad \text{and} \quad \sigma^2 = \lambda_1\Big(\sum\nolimits_i \mat{C}_i^2 \Big). \]
% Then, for any $t \geq 0,$
% \[	
%  \Prob{ \lambda_1\Big(\sum\nolimits_i \mat{G}_i\Big) \geq \mu + t } \leq d \cdot \exp\Big(- \frac{t^2/2}{\sigma^2 + At} \Big).
% \]
% \end{thm}
% 
% \end{frame}
% 
% \begin{frame}{Finishing the argument}
%  After computing $A$ and $\mat{C}_i^2$ for the summands $\mat{B}^\star \mat{x}_i \mat{x}_i^\star \mat{B},$ this gives
% \[
%  \Prob{\hat{\lambda}_k \geq \lambda_k +t} \leq (p-k+1)\cdot \exp\left( \frac{-nt^2}{32\lambda_k \sum_{i \geq k} \lambda_i } \right) \quad \text{ for } t \leq 4n \lambda_k.
% \]
% Finally, take $t = \varepsilon \lambda_k$ to see
% \[
%  \Prob{\hat{\lambda}_k \geq (1+\varepsilon) \lambda_k} \leq (p-k+1)\cdot \exp\left( \frac{-n\varepsilon^2}{32\sum_{i \geq k} \frac{\lambda_i}{\lambda_k} } \right) \quad \text{ for } \varepsilon \leq 4n.
% \]
% The proof for the case $\hat{\lambda}_k \leq \lambda_k - t$ is similar.
% \qed
% \end{frame}
% 
% \begin{frame}{Details}
% ``{\it Tail Bounds for All Eigenvalues of A Sum of Random Matrices}'', Gittens and Tropp, 2011. Preprint, \href{http://arxiv.org/abs/1104.4513}{arXiv:1104.4513}.
% \begin{itemize}
% \item Elaboration on the relative error estimation results.
% \item Similar arguments to find tail bounds for all eigenvalues of a sum of \emph{arbitrary} random matrices.
% \item An application to column subsampling.
% \end{itemize}
% 
% \end{frame}
\end{document}

